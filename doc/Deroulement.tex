\section{La conception}
La conception fut l'étape clé pour notre projet de Génie Logiciel, en effet, impossible de se lancer dans un projet d'une telle envergure sans une étape de conception.
Il a fallu tout d'abord concevoir les données que nous allions utiliser, pour cela, nous avons réaliser un diagramme de classe avec l'intégralité des données utilisable et qui vont être traitées par notre programme.
Nous avons également réfléchis à l'interface graphique et le style de jeu que nous souhaitions afin de ne pas devoir changer plus tard dans le projet.
Nous avons également réfléchis aux méthodes et aux techniques de traitements que nous allions utiliser pour nos données, celles qui serait le plus rapide, d'une part pour nous à réaliser mais aussi pour les performances de l'ordinateur.


\section{La concrétisation}

Une fois l'étape de conception finalisée, il nous fut facile de le concrétiser car le projet comporte de nombreuses parties que nous nous sommes partagé afin d'avoir un travail et une organisation efficiente.
Nous avons évidemment commencer par réaliser l'intégralité des données.
Par la suite, nous avons commencé à les traiter, dès lors que cela prenais de la forme,il fut crucial d'entamer la création d'une interface graphique afin de pouvoir avoir un meilleur rendu et affichage de notre travail.
Par la suite, plus notre projet prenais forme, plus la nécessité d'avoir un menu fut important



\section{Finalisations et améliorations}

Les finalisations d'un jeu vidéos sont innombrables,il n'y à qu'a voir les jeux d'aujourd'hui, même après leur sortie, ils requièrent des mise à jours régulières. Pour notre part, les plus grosses finalisations furent pour l'interactivité du jeu, la facilité pour l'utilisateur de comprendre et de pouvoir jouer au jeu, car de notre point de vu développeur, le jeu parait très facile à jouer car nous l'avons développé. Nous avons donc demandé à des cobayes d'essayer de jouer à Age Of Swag, et nous avons vite remarqué que le jeu était en réalité plus compliqué à comprendre pour un nouveau joueur, il nous à donc fallut rajouter une page d'aide, d'avantage de boutons, etc.

Le projet conquête est débordant d'imagination, en effet, de part son côté jeu vidéo et de notre expérience dans ce domaine, nos idées d'améliorations furent nombreuses. Toutefois par des soucis de temps et non pas de compétences, nous avons préféré nous concerter sur les plus importantes que je vais vous présenter.
Nous avons implémenté le jeux en réseau, vous pouvez dès lors jouer avec plusieurs ordinateurs sur la même partie.
Nous avons également rajouter une musique lorsque vous lancez Age Of Swag, en effet, l'absence de musique rendait le jeu moins attrayant. Afin de rendre le gameplay du jeu plus attrayant et agréable, nous avons implémenté différents types de troupes et inclut la possibilité pour le joueur de créer des bâtiments.
L'interface graphique à été également peaufiné tout au long du projet afin d'être le plus attractif et intuitif pour l'utilisateur, au delà d'une interface graphique simple, il y a toute une recherche et réflexion derrière la réalisation de l'interface, des choix des couleurs, images, etc.
